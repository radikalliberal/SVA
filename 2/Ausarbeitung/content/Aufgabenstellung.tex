\section*{Aufgabenstellung} % (fold)
\label{sec:aufgabenstellung}
Die uns gegebene Aufgabenstellung wurde sehr offen gewählt, es wurde verlangt, dass das erstellte Netz aus der ersten Übung erweitert wird in ein Timed Net. Auf Basis dieses neuen Netzes sollte sich dann eine Fragestellung überlegt werden die dann mittels Simulationen des Netzes beantwortet werden kann.

Unser Netz stellt ein Rezept und den dazugehörigen Herstellungsprozess dar. Da in diesem Prozess mehrere Abläufe parallel vollzogen werden können ist es sinnvoll zu erörtern wie schnell die Herstellung ist bei einer unterschiedliche Anzahl von Köchen. Betriebswirtschaftlich hat diese Frage Sinn da einerseits Zeit und andererseits Personalkosten ein wichtiger Faktor sind in der Gastronomie. Es ist zum einen schlecht wenn die Zubereitung einer Mahlzeit zu lange dauert, wenn sie frisch für einen Gast hergestellt wird. Und andererseits ist es schlecht wenn Köche aufeinander warten müssen da dann keine effektive Auslastung der Köche gewährleistet ist und weniger Geld erwirtschaftet wird im Verhältnis zu den Personalkosten.

Unter Berücksichtigung beider Aspekte formulieren wir folgende Fragestellung.

\begin{quote} 
\begin{tt}
\begin{center}
``Wie viele Köche müssen beschäftigt werden damit mindestens 90\% aller Gäste, 10min nach ihrer Bestellung, ihre Welfencreme bekommen?'' 
\end{center}
\end{tt}
\end{quote}
Die Arbeit in einer Küche ist natürlich wesentlich umfangreicher und wird nur zu einem kleinen Teil von unserem Petri-Netz wiedergegeben. Dieser Umstand ist bei der späteren Analyse zu berücksichtigen.





