\section*{Aufgabenstellung} % (fold)
\label{sec:aufgabenstellung}
Bei der Aufgabe des zweiten Übungstermins handelt es sich um eine Erweiterung der zuvor erhaltenen Aufgabenstellung (Übungstermin eins). Es ist gewünscht das bestehende Netz um ein Timed Net zu erweitern und sich zusätzlich eine geeignete Fragestellung zu überlegen, die mittels einer Simulation beantwortet werden kann.

Unser Netz stellt ein ein Herstellungsprozess eines Rezepts dar. In dem modellierten Prozess werden mehrere Abläufe parallel vollzogen, wodurch uns die Frage sinnvoll erscheint, wie sich die Leistung bei unterschiedlicher Anzahl an Köchen verändert. Diese Fragestellung verfolgt einerseits in dem speziellen Beispiel einen (betriebswirtschaftlichen) Sinn - sowohl die Herstellungszeit, als auch die Personalkosten spielen eine wichtige Rolle in der Gastronomie. Andererseits kann man den Prozess abstrakter als ein nebenläufiges System betrachten, welches von multiplen Prozessoren abgearbeitet wird. Wir erhoffen uns bei der Umsetzung bestimmte Muster zu erkennen, die man aus dem Bereich concurrency in der Informatik kennengelernt hat.

Kommen wir doch zunächst zurück auf die spezielle Fragestellung in der Küche. Um ein optimales Verhältnis an Köchen zu finden, muss aus unserer Sicht zwischen den folgenden beiden Aspekten abgewogen werden: Es ist zum einen schlecht wenn die Zubereitung einer Mahlzeit zu lange dauert, wenn sie frisch für einen Gast hergestellt wird. Und andererseits ist es schlecht wenn Köche aufeinander warten müssen da dann keine effektive Auslastung der Köche gewährleistet ist und weniger Geld erwirtschaftet wird im Verhältnis zu den Personalkosten.

Unter Berücksichtigung beider Aspekte formulieren wir folgende Fragestellung:

\begin{quote}
\begin{tt}
\begin{center}
``Wie viele Köche müssen beschäftigt werden damit mindestens 90\% aller Gäste, 10min nach ihrer Bestellung, ihre Welfencreme bekommen?''
\end{center}
\end{tt}
\end{quote}
Die Arbeit in einer Küche ist natürlich wesentlich umfangreicher und wird nur zu einem kleinen Teil von unserem Petri-Netz wiedergegeben. Dieser Umstand ist bei der späteren Analyse zu berücksichtigen.
